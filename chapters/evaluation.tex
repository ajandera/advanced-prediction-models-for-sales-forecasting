% !TEX root = ../thesis.tex

\chapter{Evaluation} \label{evaluation}
\section{Experiment} \label{sec:experiment}
Lets prepare the experiment to validating a long-term linear prediction model for online store orders. We will use experimentation by these steps:\\
\begin{enumerate}
    \item Define the problem and objectives: Clearly define the problem and objectives that the long-term linear prediction model is intended to solve.
    In this case, the objective is to predict online store orders over a long period of time using a linear model.
    \item Gather data: Collect historical data on online store orders, such as order dates, order amounts, and other relevant variables.
    This data will be used to train and validate the long-term linear prediction model. Our data was collect from online system storePredictor
    which is available on storepredictor.com and data are anonymized and pseudonimized to be used for next calculations.
    \item Prepare the data: Clean and preprocess the data to ensure that it is consistent and suitable for use in the long-term linear prediction model.
    This may involve tasks such as removing duplicates, handling missing values, and transforming variables as needed.
    Preprocessing of our data is described in \ref{subsec:preprocessing}
    \item Develop the long-term linear prediction model: Using the historical data, develop a long-term linear prediction model that can
    forecast online store orders over a specified time period, which is detailed described in \ref{sec:extlonglp} and practically applied in \label{subsec:calculate_models}
    \item Validate the model: Once the model is developed, it needs to be validated to ensure that it is accurate and effective.
    This can be done by comparing the model's predictions with actual online store orders over a specified time period, described in \label{subsec:validatibg_models}
    \item Evaluate the results: Analyze the results of the experiment to determine the accuracy of the long-term linear prediction model.
    This will involve calculating metrics such as the mean square error (MSE), r-squared ($R^2$) and the root mean square error (RMSE) \ref{subsec:experimentResults}.
    \item During solving previous steps the model was redefined and revalidate when the results are not satisfactory, refine the model and repeat the validation
    process until an accurate and effective long-term linear prediction model is developed.
\end{enumerate}
In summary, to prepare an experiment to validate a long-term linear prediction model for online store orders, you need to
define the problem and objectives, gather and prepare the data, develop the model, validate it, evaluate the results, and refine and revalidate
the model if necessary.
    \subsection{Preprocessing of input data} \label{subsec:preprocessing}
    Preprocessing the data is an important step in preparing the data for a linear order prediction model. Here are some common
    steps to preprocess the data for linear order prediction:

    \begin{enumerate}
        \item Data cleaning: Remove any irrelevant data or duplicate records in the dataset. In our purpose is to clean the unsuccessfull orders,
        returned orders and fraud orders from competitors to detect power of the store.
        \item Handling missing data: for fill in any missing values in the dataset machine learning algorithms K-Nearest Neighbors (KNN) \ref{sec:knn} will be used
        to check the dataset.
        \item Feature selection: Select the relevant features (predictors) that are likely to have a strong influence on the order
        prediction. This may include variables such as customer demographics, purchase history, product attributes, and marketing campaigns.
        \item Feature scaling: Scale the features so that they are on the same scale to ensure that each feature has equal importance.
        This can be done using normalization or standardization techniques.
        \item Handling categorical variables: Convert categorical variables into numerical values using techniques
        such as one-hot encoding, ordinal encoding, or label encoding.
        \item Dimensionality reduction: If the dataset contains many features, use dimensionality reduction
        techniques such as principal component analysis (PCA) or linear discriminant analysis (LDA) to reduce the number of features and simplify the model.
        \item Handling outliers: Detect and handle any outliers in the dataset using appropriate techniques
        such as Z-score, Tukey’s method, or machine learning algorithms such as Isolation Forest or Local Outlier Factor.
    \end{enumerate}
    Overall, the goal of this steps for linear order prediction experiment is to ensure that the data is clean, complete, and properly
    formatted for use in the linear prediction model. This helps to improve the accuracy and effectiveness of the model in predicting online store orders.
    \subsection{Model for sales forecasting} \label{subsec:calculate_models}
    \subsection{Validation model} \label{subsec:validatibg_models}
    \subsection{Results} \label{subsec:experimentResults}
