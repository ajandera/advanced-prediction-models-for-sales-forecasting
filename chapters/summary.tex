% !TEX root = ../thesis.tex

\chapter{Summary} \label{summary}

\chapter{Resume} \label{resume}
    \section{Analyza} \label{sk:analytic}
    Lineárna predikcia je štatistická metóda používaná na predpovedanie budúcich hodnôt na základe historických dát pre identifikaciu parametrov pouziva
    Durbin-Levinsonov algoritmus coz je metóda riešenia lineárnej predikcie pre autoregresívne (AR) modely1, ktoré sú modely, kde aktuálny výstup
    závisí od predchádzajúcich výstupov. Algoritmus rieši problém lineárnej predikcie nájdením koeficientov AR modelu, ktoré minimalizujú chybu predikcie.
    Výsledné AR koeficienty môžu byť použité na predpovedanie budúcich hodnôt na základe minulých pozorovaní. Táto metóda by sa mala
    používať s použitím vzoru lineárneho vzťahu medzi nezávislými a závislými premennými. Tu je základný prehľad krokov pri použití lineárnej
    predikcie na predpovedanie dát o predaji:\\
    \\
    \begin{enumerate}
    \item Zbierajte dáta o predaji: Získajte historické dáta o predaji produktu alebo služby, ktorú chcete predpovedať.
    \item Vykreslite dáta: Vykreslite dáta o predaji v čase, aby ste vizuálne preskúmali trend a identifikovali akékoľvek vzory.
    \item Vyberte model: Vyberte vhodný lineárny model na zobrazenie vzťahu medzi nezávislými a závislými premennými v dátach.
    Napríklad by ste si mohli vybrať jednoduchý lineárny regresný model.
    \item Natrénujte model: Natrénujte vybraný model na historických dátach o predaji pomocou metódy ako najmenšie štvorce.
    \end{enumerate}
    Autoregresívne (AR) modely sú modely časových radov, ktoré popisujú vzťah medzi súčasnou hodnotou premennej a jej minulými hodnotami.
    V autoregresívnom modeli je každá pozorovanie modelované ako lineárna kombinácia minulých pozorovaní, s váhami nazývanými AR koeficienty.
    AR modely sú široko používané v rôznych oblastiach, ako je ekonomika, inžinierstvo a financie, na modelovanie a predpovedanie časových radových dát.
    Poradie AR modelu, označované ako "p", sa vzťahuje na počet minulých hodnôt použitých na predpovedanie súčasnej hodnoty.
    Napríklad AR(1) model používa len predchádzajúce pozorovanie na predpovedanie súčasnej hodnoty, zatiaľ čo AR(2) model používa predchádzajúce
    dve pozorovania.\\
    \\
    Vykonajte predpovede: Použite trénovaný model na predpovedanie budúcich predajových dát. Môžete chcieť generovať predpovede na niekoľko
    mesiacov alebo rokov dopredu.
    \begin{enumerate}
        \item Zhodnoťte model: Posúďte presnosť predpovedí porovnaním s faktickými predajovými dátami. Použite metriky, ako je priemerná absolútna
        chyba alebo koreňová stredná štvorcová chyba, na kvantifikáciu výkonu modelu.
        \item Vylepšite model: Ak je to potrebné, vylepšite model pridaním ďalších nezávislých premenných alebo transformáciou existujúcich premenných.
        \item Opakujte kroky trénovania a hodnotenia, kým nebudete mať model, ktorý poskytuje presné prognózy.
        \item Pre výpočet posunu v dlhodobom predpovedaní sa môže použiť autokorelačná metóda. V tejto práci sa vyvinie neurónová sieť na identifikáciu
        posunu a podobný mechanizmus pre optimálne určenie poradia.
    \end{enumerate}
    \textbf{Modely používané pre predajové dáta} \\
    Existuje niekoľko matematických modelov používaných pre predpovedanie predaja, vrátane:\\
    \begin{enumerate}
        \item Modely časových radov: Tieto modely sa používajú na analýzu a predpovedanie predajových dát v čase, ako sú sezónne vzorce,
        trendy a fluktuácie. Príklady zahŕňajú ARIMA (AutoRegressive Integrated Moving Average), SARIMA (Seasonal ARIMA) a exponenciálne vyhladzovanie.
        \item Regresné modely: Tieto modely používajú historické údaje na určenie vzťahu medzi predajom a jednou alebo viacerými
        nezávislými premennými, ako sú cena, propagácia a reklama. Príklady zahŕňajú lineárnu regresiu, logistickú regresiu a viacnásobnú regresiu.
        \item Modely stromových rozhodnutí: Tieto modely používajú štruktúru stromu na rozhodovanie založené na vzťahu medzi
        predajom a viacerými nezávislými premennými. Príklady zahŕňajú CART (Klasifikácia a regresia stromu) a náhodný les.
        \item Modely strojového učenia: Tieto modely používajú algoritmy ako neurónové siete a stroje s podpornými
        vektormi na predikovanie na základe vzorov v údajoch.
    \end{enumerate}
    \textbf{Neurónove sieťe} \\
    Neurónová sieť je druh algoritmu strojového učenia inšpirovaný štruktúrou a funkciou biologických neurónov v ľudskom mozgu.
    Skladá sa z prepojených uzlov, nazývaných neuróny, ktoré sú usporiadané do vrstiev. Vstupná vrstva prijíma surové dáta, ako sú obrázky alebo
    text, a prenáša ich do skrytých vrstiev, ktoré vykonávajú výpočty a váhy sa aplikujú na vstupné dáta pre vytvorenie predikcie.
    Nakoniec výstupná vrstva produkuje konečnú predikciu alebo klasifikáciu.\\
    \\
    Ako môžete vidieť na obrázku\ref{fig:perceptron}, každý vstup $Xn$ by mal byť správne ohodnotený určitou váhou $W$ n predtým,
    než všetky signály vstúpia do sumovacej fázy. Potom sa vážené súčty prenášajú do aktivačnej jednotky produkujúcej výstupný signál neurónu.\\
    \\
    Neurónové siete sa trénujú na veľkých dátových súboroch pomocou procesu nazývaného spätné šírenie chyby, ktorý upravuje váhy a sklon neurónov,
    aby minimalizoval rozdiel medzi predpovedaným výstupom a skutočným výstupom. Akonáhle je neurónová sieť natrénovaná, môže sa použiť na predpovedanie
    nových dát.\\
    \\
    Neurón je základnou stavebnou jednotkou neurónovej siete, známy aj ako umelej neurón alebo perceptrón. Modeluje sa podľa biologického
    neurónu v ľudskom mozgu, ktorý prijíma vstupné signály z iných neurónov, spracováva ich a posiela výstupné signály do ďalších neurónov.\\
    \\
    V neurónovej sieti neurón prijíma vstup od iných neurónov alebo priamo od vstupných dát, aplikuje na vstup matematickú funkciu a produkuje
    výstup, ktorý sa posiela do ďalších neurónov v sieti. Vstupom do neurónu je zvyčajne vektor čísel a každý vstup sa násobí príslušnou váhou.\\
    \\
    Potom neuron sčíta vážené vstupy, pridáva sklon a aplikuje aktivačnú funkciu na výsledok. Úlohou aktivačnej funkcie je zaviesť nelinearitu do
    neurónu, čo umožňuje neuronovej sieti naučiť sa zložité vzorce a vzťahy v dátach. Existuje niekoľko rôznych typov aktivačných funkcií, ktoré
    sa môžu použiť, ako napríklad sigmoidná funkcia, ReLU (Rectified Linear Unit) funkcia a tanh (hyperbolická tangens) funkcia.\\
    \\
    Výstup neurónu sa zvyčajne posúva do ďalších neurónov v nasledujúcej vrstve neurónovej siete. Váhy a sklon neurónov sa počas trénovania
    prispôsobujú technikou spätného šírenia chyby, ktorá zahŕňa výpočet gradientu chyby vzhľadom na váhy a aktualizovanie ich pomocou
    optimalizačného algoritmu, ako je stochastický gradientový zostup.\\
    \\
    Celkovo neuróny v neurónovej sieti spolupracujú na učení vzorcov a vzťahov vstupných dát a produkujú výstup, ktorý sa môže použiť pre rôzne
    úlohy, ako je klasifikácia, regresia a predikcia.\\
    \\
    Neurónové siete sa úspešne uplatňujú v širokej škále oblastí, vrátane rozpoznávania obrazov a reči, spracovania prirodzeného
    jazyka a autonómnych vozidiel, medzi inými.
    \section{Synteza}
    Vychadzajme z analytickej časti \ref{sk:analytic}, vytvorme nové matematické modely a prístupy, aby sme mohli vykonať rýchle a presné predpovede predaja,
    ktoré sa skladajú z dlhodobej lineárnej predikcie s individuálnymi váhami vypočítanými pre každé obdobie, založené na Levinson-Durbinovej schéme
    nazývanej Extended Linear Prediction (ELP). Očakávame lepšie výsledky než pri použití predikcie založenej na krátkodobej alebo
    dlhodobej štandardnej lineárnej predikcii (viď časť 1.4). Nakoniec, náš prístup bude vracať budúce hodnoty pre predaj spoločností na základe
    predchádzajúcich dát s lepšou odchýlkou, než to dokáže lineárna predikcia.\\
    \\
    Pre vytvorenie matematického modelu na predikciu predajných dát s periodickými trendmi môžete použiť model sezónnej ARIMA (SARIMA).
    Tento model zohľadňuje sezónne variácie v dátach a používa autoregresívne a kĺzavé priemerové členy na zachytenie vzorov a trendov v dátach.\\
    \\
    Pre svoj účel som vytvoril rozšírenú dlhodobú predikciu, ktorá sa vysporiada so sezónnymi a opakujúcimi sa vzormi v ekonomických dátach.
    Tento korekčný mechanizmus sa používa na zohľadnenie historických vrcholov v grafoch cez dataset. Táto jednoduchá korekcia zvyšuje
    presnosť modelu a získava lepšiu odpoveď vďaka psychologickým, sociologickým a marketingovým aspektom v datasete.\\
    \\
    Na nastavenie periodických váh je potrebné vytvoriť vektor korekčných parametrov z pôvodného datasetu pomocou štatistických parametrov
    mediánu a štandardnej odchýlky z datasetu. Ako základnú rovnicu použijeme rovnicu pre dlhodobú predikciu 4.1 s novými váhami a získame tak lepšie výsledky.
    \section{Experiment}
    Pripravme experiment na overenie dlhodobého lineárneho predikčného modelu objednávky internetového obchodu. Použijeme experimentovanie podľa týchto krokov:
    \begin{enumerate}
    \item Definujte problém a ciele: Jasne definujte problém a predmet ktoré má vyriešiť dlhodobý lineárny predikčný model. V tomto
    Cieľom je predpovedať objednávky v internetovom obchode na dlhé časové obdobie pomocou lineárneho modelu.
    \item  Zhromažďovanie údajov: Zhromažďujte historické údaje o objednávkach online obchodov, ako je napríklad objednávka
    dátumy, sumy objednávok a ďalšie relevantné premenné. Tieto údaje budú použité trénovať a overovať model dlhodobej lineárnej predikcie. Naše údaje boli
    zbierať z online systému storePredictor, ktorý je dostupný na storepredictor.com a údaje sú anonymizované a pseudonimizované,
    aby sa mohli použiť na ďalšie účely výpočty.
    \item  Pripravte údaje: Vyčistite a predspracujte údaje, aby ste sa uistili, že sú konzistentné stan a vhodné na použitie v dlhodobom
    lineárnom predikčnom modeli. Toto môže zahŕňajú úlohy, ako je odstraňovanie duplikátov, spracovanie chýbajúcich hodnôt a transformovať premenné
    podľa potreby. Predspracovanie našich údajov je popísané v ref{subsec:preprocessing}.
    \item  Vytvorte model dlhodobej lineárnej predikcie: Pomocou historických údajov, vyvinúť dlhodobý lineárny predikčný model,
    ktorý dokáže predpovedať internetový obchod objednávky počas určitého časového obdobia, ktoré je podrobne popísané v bode 4.4 a prakticky
    aplikovaný v 
    \item  Overenie modelu: Keď je model vytvorený, je potrebné ho overiť aby sa zabezpečilo, že je presný a účinný.
    \item Vyhodnoťte výsledky: Analyzujte výsledky experimentu, aby ste určili presnosť modelu dlhodobej lineárnej predikcie. To bude zahŕňať
    výpočet metrík, ako je stredná štvorcová chyba (MSE), r-kvadrát (R2) a stredná kvadratická chyba (RMSE) 5.1.3.
    \item Počas riešenia predchádzajúcich krokov bol model predefinovaný a revalidovaný, keď výsledky nie sú uspokojivé, upravte model a zopakujte validáciu
    až kým nevznikne presný a efektívny dlhodobý lineárny predikčný model vyvinuté.
    \end{enumerate}
    Stručne povedané, pripraviť experiment na overenie dlhodobej lineárnej predpovede model pre objednávky v internetovom obchode, musíte definovať
    problém a ciele, zhromažďovať a pripravovať údaje, vyvíjať model, overovať ho, hodnotiť výsledky, a ak je to potrebné, model spresnite a znovu overte.
    \section{Zaver}