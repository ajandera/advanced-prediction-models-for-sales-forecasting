% !TEX root = ../thesis.tex

\chapter{Analytická časť}

Analytická časť záverečnej práce analyzuje existujúce podobné prístupy k~riešeniu stanoveného problému. Autor práce musí uviesť v~tejto časti existujúce prístupy a riešenia, pričom musí zaujať stanovisko k~týmto prístupom a riešeniam a opísať ich výhody a nedostatky. Prevažne v~tejto časti autor používa odkazy na použité zdroje. Autor v~analýze nepreberá odseky z~cudzích prác ale uvádza prevažne vlastné postoje podložené odkazmi na literatúru. Analytická časť práce by teda nemala byť len povrchným prepisom základných informácií z~Wikipédie alebo zo stránok opisovaných nástrojov. Je potrebné aby bola analýza podporená aj experimentmi ak to umožňuje téma práce (napr. vyskúšam softvér). Vďaka popisu existujúcich riešení autor pochopí problematiku, viac sa nad riešeniami zamyslí, usporiada si ich, zistí ich kladné a záporné vlastnosti, z~čoho potom postupne vyplynie návrh vlastného riešenia v~syntetickej časti. Analytická časť tvorí zvyčajne ¼ jadra práce.

Analytickú časť je možné rozdeliť na niekoľko kapitol, ktoré budú venované rôznym analyzovaným témam. Názvy kapitol majú zodpovedať tomu, čo je v~kapitole opisované. Napríklad ak v~práci analyzujete súčasný stav v~oblasti medzigalaktických letov, namiesto všeobecného názvu "`Analýza súčasného stavu"' by mal byť použiťý názov analyzovanej témy --- "`Medzigalaktické lety"'.


\blindtext

\blindtext
