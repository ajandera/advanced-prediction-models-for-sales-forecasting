% !TEX root = ../thesis.tex
\chapter{Goal of the thesis} \label{sec:goal}
In this thesis we are focus on improving linear prediction in sales and financial forecasting and combine it with
modern machine learning approaches. Machine learning techniques are used to improve the accuracy of linear
prediction models. Specifically, there are two ways in which machine learning are applied to linear prediction:
\begin{enumerate}
    \item Feature engineering: In this approach, machine learning is used to extract relevant features from the input
    signal that can be used as input to a linear prediction model.The extracted features can capture complex patterns
    in the data that are not captured by the raw input. Feature engineering can be done using techniques such as
    principal component analysis, wavelet transform, and Fourier transform.
    \item Model selection and training: In this approach, machine learning is used to select the best linear
    model for the prediction task and to estimate its parameters from the data. This can involve selecting the best
    set of input variables for the linear model, choosing the best regularization parameter to avoid overfitting, and
    optimizing the model hyperparameters. Common machine learning algorithms used for linear prediction include linear
    regression, support vector regression, and artificial neural networks.
\end{enumerate}
The goal of long-term linear prediction is to estimate future values of a signal or time series based on its past
values using a linear model.
The linear prediction model uses a set of coefficients to weight past values of the signal and produce a prediction for
future values. The accuracy of the prediction depends on the quality of the model and the complexity of the
underlying signal. Combine this principles get to us the best results from linear prediction over
sales and financial datasets.
\chapter{Methodology} \label{sec:methodology}
    \section{Characteristics of the research object} \label{subsec:research_object}
    Sales data and financial datasets are two broad categories of data that have different characteristics and are used
    for different purposes. Here are some general characteristics of each type of dataset:\\
    \\
    \textbf{Sales data}\\
    Typically contains transactional information, such as the date, time, location, and amount of a purchase
    Can include additional information about the customer, such as their demographic profile, purchase history, and
    preferences Often analyzed to understand customer behavior, such as buying patterns, trends, and preferences
    May have seasonal or cyclical patterns, depending on the nature of the product or service being sold
    Can be used to optimize marketing and sales strategies, such as targeting specific customer segments, promoting
    certain products, or adjusting prices and discounts.\\
    \\
    \textbf{Financial datasets}\\
    Typically contains financial information, such as the revenue, expenses, assets, liabilities, and cash
    flow of a company or organization can include additional information about the market, such as interest rates,
    exchange rates, and stock prices. Often analyzed to evaluate the financial performance and health of a company,
    such as profitability, solvency, and liquidity. May have regulatory or compliance requirements, such as financial
    reporting standards or tax laws. Can be used to make strategic decisions, such as investment, merger and
    acquisition, or divestiture. Both sales data and financial datasets can be used for forecasting and modeling,
    but they have different analytical techniques and tools. Sales data often requires customer segmentation,
    predictive analytics, and machine learning algorithms, while financial datasets require financial ratio analysis,
    time series forecasting, and risk assessment.
    \section{Methods} \label{subsec:methods}
        \subsection{Linear regression} \label{sec:linear}
        Linear regression~\cite{linear} attempts to model the relationship between two variables by fitting a linear
        equation to observed data. One variable is considered to be an explanatory variable, and the other is
        considered to be a dependent variable. For example, we want to relate the weights of individuals to their
        heights using a linear regression model. Before attempting to fit a linear model to observed data, we should
        first determine if there exists a relationship between the variables of interest. This does not necessarily
        mean that one variable causes the other, but that there is some significant association between them.
        To determine the strength of relationship a scatterplot can be a helpful tool.
        If there appears to be no association between the proposed explanatory and dependent variables, then
        fitting a linear regression model to the data probably will not provide a useful model.
        A valuable numerical measure of association between two variables is the correlation coefficient.
        That is a value from $[-1, 1]$ range indicating the strength of the association of the observed data
        for the two variables.\\
        A linear regression line has an equation of the form $Y = a + bX$, where $X$ is the explanatory variable
        and $Y$ is the dependent variable. The slope of the line is $b$, and $a$ is the intercept.
        Example of some basic linear models:\\
        \begin{equation} \label{eq:4}
        \begin{array}{l@{}l}
            Y = ax + b\\
            Y = a + bx + c\\
            Y = a\sin x + b\\
        \end{array}
        \end{equation}
        \subsection{Linear prediction} \label{subsec:lp}
        Linear prediction is a statistical technique used to forecast future values based on past observations.
        It is a method for modeling the relationship between a dependent variable and one or more independent
        variables in a linear form. The goal of linear prediction is to find the best linear approximation of the
        relationship between the variables, which can then be used to make predictions about future
        values of the dependent variable.
        \subsection{Backpropagation} \label{subsec:lp}
        Backpropagation is a widely used algorithm for computing the gradients of the loss function with
        respect to the weights of a neural network. It is the backbone of training deep neural networks using
        stochastic gradient descent (SGD) or its variants.
        The backpropagation algorithm computes the gradient of the loss function with respect to each
        weight in the network by recursively applying the chain rule of differentiation.
        The algorithm is typically implemented in two phases:
        \begin{enumerate}
            \item Forward pass: The forward pass involves computing the output of each layer in the network,
            starting from the input layer and propagating through the hidden layers to the output layer.
            The output of each layer is computed as a function of the input to the layer and the weights of the layer.
            The forward pass computes the predictions of the network on a given input.
            \item Backward pass: The backward pass involves computing the gradients of the loss function with
            respect to each weight in the network, starting from the output layer and propagating backwards
            through the hidden layers to the input layer. The gradients are computed by applying the chain rule of
            differentiation to the output of each layer. The gradients are then used to update the weights of
            the network using an optimization algorithm such as SGD.
        \end{enumerate}
        The backpropagation algorithm can be optimized using various techniques, such as parallel computing, weight
        sharing, and regularization. It is a powerful tool for training deep neural networks with many layers and
        millions of parameters, and has enabled significant advances in many areas of machine learning, including
        computer vision, natural language processing, and speech recognition.
        \section{Datasets} \label{subsec:datasets}
        Creating a science research dataset involves several steps, which may vary depending on the nature of the
        research and the specific field of science. Here are some general steps to consider:
        \begin{enumerate}
            \item Determine the research question: The first step in creating a research dataset is to determine the
            research question. This will help we identify the data we need to collect and the type of dataset
            we need to create.
            \item Choose the data sources: Once we have identified the research question, we need to determine
            the data sources we will use to create our dataset. Depending on the research question, we may use
            publicly available data, data from surveys or experiments, data from literature, or a combination of these.
            \item Collect and organize the data: Collecting data can involve different methods, such as surveys,
            experiments, literature reviews, or data mining. We need to make sure that the data we collect is reliable,
            accurate, and relevant to we research question. Once you have collected the data, we need to organize
            it in a way that is easy to analyze.
            \item Clean and preprocess the data: Before we can analyze the data, we need to clean and preprocess it.
            This involves removing any errors, missing values, or duplicates in the data. We may also need to
            transform or normalize the data to make it compatible with the analysis methods we plan to use.
            \item Perform exploratory analysis: Once the data is cleaned and preprocessed, we can perform exploratory
            analysis to identify patterns, trends, or relationships in the data. This can help us refine our research
            question or identify areas that require further investigation.
            \item Perform statistical analysis: Depending on the research question, we may need to perform statistical
            analysis to test hypotheses or evaluate the significance of the results. This can involve using regression
            analysis, hypothesis testing, or other statistical methods.
        \end{enumerate}
        Validate the dataset: After we have analyzed the data, we need to validate the dataset to ensure that the
        results are accurate and reliable. This can involve comparing the results to other datasets or conducting
        further experiments to verify the results. Overall, creating a science research dataset involves careful
        planning, data collection, preprocessing, and analysis to ensure that the data is accurate, reliable, and
        relevant to the research question.
    \section{Comparison criteria} \label{subsec:comparison}
    Finally we define the method to compare our models results.
    Absolute number of income value prediction should not be important for the store owners because of that we
    calculated the aberration for each month prediction and then we easily calculate quarterly and yearly results.
    The sum of squared errors (SSE), defined by:
    $$SSE = \sum^n_{i=1}w_i(y_i - \overline{y_i})^2,$$
    between the fitting models and the used data serves as the fitting criterion,
    with values closer to $0$ indicating a smaller random error component of the model.
    Also some other quality measures were evaluated, \textit{i.e.} the R-square from interval $[0,\ 1]$,
    that indicates the proportion of variance satisfactory explained by the fitting-model (\textit{e.g.}
    R-square $= 0.7325$ means that the fit explains $73.25\%$ of the total variation in the data about the average);
    R-square is defined as the ratio of the sum of squares of the regression (SSR) and the total sum of squares (SST).
    SSR is defined as
    $$SSR = \sum_{i=1}^nw_i(\overline{y_i} - \overline{y_i})^2.$$
    SST is also called the sum of squares about the mean, and is defined as
    $$SST = \sum_{i=1}^nw_i(y_i - \overline{y})^2,$$
    where SST = SSR + SSE. Givenm these definition, R-square is expressed as
    $$\frac{SSR}{SST} = 1 - \frac{SSE}{SST}.$$
    The adjusted R-square statistic, with values smaller or equal to $1$, where values closer to $1$ indicate a better
    fit; the root mean squared error (RMSE):\\
    $$RMSE = s = \sqrt{\frac{SSE}{v}}$$
    with values closer to $0$ indicating a fit more useful for prediction.\\
    \\
    Mean square error (MSE) is a commonly used metric to measure the average squared difference between the actual and
    predicted values of a continuous variable. It is a measure of how well a prediction model performs in predicting
    continuous outcomes.\\
    To calculate the MSE, we first take the difference between the predicted and actual values for each data point,
    square these differences, and then take the average of the squared differences across all data points.\\
    \\
    The formula for calculating MSE is:

    \begin{equation}
    MSE = \frac{1}{n}\sum_{i=1}^{n}(y_i - \hat{y}_i)^2,
    \end{equation}
    \\
    where n is the total number of data points, $y_i$ is the actual value of the $i-th$ data point, and $\hat{y}_i$ is
    the predicted value of the $i-th$ data point.\\
    \\
    The square of the difference in each data point ensures that the errors are positive and provides a way to measure
    the magnitude of the error. By squaring the errors, the MSE gives more weight to larger errors and less weight to
    smaller errors.\\
    \\
    MSE is a popular metric used in many fields, such as machine learning, statistics, and engineering, to evaluate the
    performance of prediction models. A lower MSE indicates better performance of the model in predicting the outcomes.

    \section{Statistics methods} \label{subsec:statistics}
    \textbf{Median}\\
    The median is a statistical measure that represents the middle value of a dataset. It is a value that separates
    the dataset into two halves: half of the values are greater than the median, and half of the values are less than
    the median. In other words, the median is the value that is exactly in the middle of the dataset when the
    values are arranged in order of magnitude.
    \\
    To compute the median of a dataset, we first sort the values in ascending or descending order. If the dataset
    has an odd number of values, the median is the middle value. For example, in the dataset ${1, 2, 3, 4, 5}$, the
    median is $3$. If the dataset has an even number of values, the median is the average of the two middle values.
    For example, in the dataset ${1, 2, 3, 4}$, the median is $(2 + 3) / 2 = 2.5$.
    \\
    The median is a robust statistic, meaning that it is not affected by outliers or extreme values in the dataset,
    unlike the mean. This makes it a useful measure of central tendency in datasets with a large number of outliers or
    skewed distributions. The median is commonly used in various applications, such as finance, economics, and social
    sciences, to summarize and compare datasets.\\
    \\
    \textbf{Standard deviation}\\
    Standard deviation is a statistical measure that quantifies the amount of variation or dispersion in a dataset.
    It measures how far the values in a dataset deviate from the mean, or average, of the dataset. The standard
    deviation is a non-negative number and has the same units as the data being measured.
    \\
    To compute the standard deviation of a dataset, we first calculate the mean of the dataset.
    Then, we calculate the difference between each value in the dataset and the mean, square each difference, and
    sum up the squared differences. Finally, we divide the sum of squared differences by the number of values in the
    dataset, and take the square root of the result. This gives us the standard deviation of the dataset.
    \\
    A small standard deviation indicates that the values in the dataset are tightly clustered around the mean,
    while a large standard deviation indicates that the values are widely spread out from the mean. Standard
    deviation is commonly used in various applications, such as finance, engineering, and natural sciences, to
    analyze and compare datasets. It is also an important parameter in many statistical tests and models, such as the
    normal distribution and the t-test.

\chapter{Syntactic part} \label{sec:syntactic}
Based on the Analytical part~\ref{sec:analytical} let us create new mathematical models and approaches to made a fast
and accuracy sales forecasting consist of long-therm linear prediction with individual weights calculated for each
period all based on Levinson-Durbin scheme caled Extended linear prediction (ELP)
We expect to get better results than by using prediction based on short-term or long-term standard linear
prediction (see section~\ref{sec:lp}).  Finally, our approach will return future values for sales companies
based on previous data with better aberration than linear prediction has.
    \section{Order calculation} \label{sec:ordercalc}
        \textbf{Create neural site for order prediction}\\
        \\
        Creating a neural network for linear prediction and optimal order detection involves several steps.
        Here's a general overview of the process:
        \begin{enumerate}
            \item Data Preparation: Collect a dataset of input/output pairs that represent the relationship we want
            the neural network to learn. For linear prediction, this could be a time-series dataset where we want to
            predict the next value based on the previous values. For optimal order detection, this could be a dataset
            where we have inputs and the corresponding optimal order.
            \item Data Preprocessing: Preprocess the dataset by normalizing the inputs and outputs to a common range,
            shuffling the dataset, and splitting it into training, validation, and testing sets.
            \item Model Architecture: Choose an appropriate neural network architecture for our problem. For linear
            prediction, we could use a recurrent neural network (RNN) such as LSTM or GRU. For optimal order
            detection, we could use a feedforward neural network.
            \item Training: Train the neural network on the training set using an appropriate optimization algorithm
            such as stochastic gradient descent (SGD), Adam, or RMSProp. During training, monitor the performance of the
            network on the validation set to detect overfitting and adjust the hyperparameters accordingly.
            \item Testing: Evaluate the performance of the trained neural network on the testing set
            to get an estimate of its generalization ability.
            \item Deployment: Deploy the trained neural network to make predictions on new data.
        \end{enumerate}
        \textbf{Choose activation function}\\
        \\
        Choosing the right activation function for linear prediction order detection in neural networks depends on
        several factors such as the type of data, the complexity of the problem, and the architecture of the neural
        network. Here are some guidelines to help we choose the right activation function:
        \begin{enumerate}
            \item Linear activation function: If we want to perform linear regression or linear prediction, we can
            use a linear activation function. The output of a linear activation function is proportional to the input,
            which makes it ideal for linear regression problems.
            \item ReLU activation function: The Rectified Linear Unit (ReLU) activation function is one of the most
            popular activation functions used in deep learning. ReLU is simple to compute and provides good results
            for most classification and regression problems.
            \item Sigmoid activation function: The Sigmoid activation function maps any input value to a value
            between 0 and 1. It is commonly used in binary classification problems, where the goal is to classify
            data into two categories.
            \item Tanh activation function: The hyperbolic tangent (Tanh) activation function is similar to the sigmoid
            function but maps input values to a range between -1 and 1. It is also used in binary classification problems.
            \item Softmax activation function: The Softmax activation function is used in multi-class classification
            problems. It maps input values to a probability distribution over the output classes.
            \item In general, it's a good idea to start with the ReLU activation function and see if it works well
            for our problem. If not, we can try other activation functions that are appropriate for our problem.
            It's also important to experiment with different architectures and hyperparameters to find the
            best model for our data.
        \end{enumerate}
        \textbf{Train neural site}\\
        \\
        Training a neural network involves several steps:
        \begin{enumerate}
            \item Preparing the data: This involves collecting and preprocessing the data that will be used to train
            the neural network. The data needs to be cleaned, normalized, and split into training, validation, and
            testing sets.
            \item Defining the architecture: This involves selecting the number and type of layers, the activation
            functions, the optimization algorithm, and the loss function. The architecture should be chosen
            based on the problem being solved.
            \item Initializing the weights: The weights of the neural network are initialized randomly.
            \item Forward propagation: The input data is fed into the neural network, and the outputs of each layer are
            computed using the activation functions.
            \item Backward propagation: This involves computing the gradient of the loss function with respect to the 
            weights of the neural network. The gradients are computed using the chain rule of differentiation.
            \item Updating the weights: The weights of the neural network are updated using an optimization algorithm
            such as stochastic gradient descent.
            \item Testing the model: Once the model is trained, it needs to be tested on a separate dataset to
            evaluate its performance.
            \item Fine-tuning the model: Based on the performance of the model, adjustments can be made to the
            architecture, hyperparameters, or training data to improve its performance.
            \item The above steps are repeated multiple times until the performance of the model
            reaches a satisfactory level. It's important to note that training a neural network is a computationally
            intensive task, and it may take a significant amount of time and resources.
        \end{enumerate}
        \textbf{Run neural site to get optimal order}\\
        \\
        Once a neural network is trained, there are several steps we can take to get the best performance out of it:
        \begin{enumerate}
            \item Use the validation set to tune hyperparameters: The validation set is a dataset that is separate
            from the training and testing sets and is used to tune hyperparameters such as learning rate, batch size,
            and number of epochs. By experimenting with different hyperparameters, we can find the optimal settings
            that result in the best performance.
            \item Test the model on a separate dataset: Once the model is trained and the hyperparameters are tuned,
            it's important to test the model on a separate dataset to evaluate its performance. This dataset should be
            completely separate from the training and validation sets.
            \item Use data augmentation techniques: Data augmentation techniques such as rotation, translation, and
            scaling can be used to generate additional training data, which can improve the performance of the model.
            \item Use ensembling techniques: Ensembling techniques such as bagging and boosting can be used to
            combine the predictions of multiple models to improve the overall performance.
            \item Use regularization techniques: Regularization techniques such as L1 and L2 regularization
            can be used to prevent overfitting and improve the generalization of the model.
            \item Fine-tune the model on new data: If new data becomes available, the model can be fine-tuned on the
            new data to further improve its performance.
            \item By following these steps, we can get the best performance out of a trained neural network.
            It's important to remember that there is no single approach that works best for all problems, and
            experimentation is often necessary to find the optimal solution.
        \end{enumerate}
    \section{Shift calculation} \label{sec:shiftcalc}
        \subsection{Autocorrelation} \label{subsec:acorr}
        \begin{enumerate}
            \item Compute the autocorrelation function of the signal using the xcorr function in MATLAB.
            The xcorr function returns the cross-correlation sequence of the signal with itself.\\
            \\
            $[Rxx, lag] = xcorr(x, maxlag);$\\
            \\
            where x is the input signal and maxlag is the maximum lag to compute. Rxx is the cross-correlation
            sequence, and lag is the corresponding lag vector.
            \item Find the index of the maximum value in the autocorrelation function. We can use the max function to
            find the maximum value and its index.\\
            \\
            $[~, idx] = max(Rxx);$\\
            \\
            The ~ ignores the value of the maximum, and idx is the index of the maximum value in the Rxx vector.
            \item Calculate the shift value from the lag index. The shift value is simply the lag value
            corresponding to the maximum value in the autocorrelation function.\\
            \\
            $shift = lag(idx);$\\
            \\
            The shift value is the lag at which the autocorrelation function has its maximum value, which
            corresponds to the shift value for long-term linear prediction. Note that the shift value is expressed in
            samples, so we may need to convert it to a time delay if our signal has a specific sampling rate.
        \end{enumerate}
        \subsection{Neural network} \label{subsec:neural}
        \textbf{Create neural site for shift and long shift prediction}\\
        \\
        To create a neural network for shift and long shift prediction, we can follow these steps:
        \begin{enumerate}
            \item Define the problem: The first step is to define the problem we want to solve. In this case, the
            problem is to predict the shift and long shift of a time series data.
            \item Collect and preprocess the data: Collect the data that we want to use for training and
            testing the neural network. Preprocess the data by normalizing it and splitting it into training,
            validation, and testing sets.
            \item Choose the architecture: Choose the architecture for our neural network based on the problem
            we want to solve. In this case, a recurrent neural network (RNN) architecture such as a Long Short-Term
            Memory (LSTM) or Gated Recurrent Unit (GRU) may be suitable, as they are specifically designed to handle
            sequential data.
            \item Define the input and output: Define the input and output of our neural network. The input should
            be the time series data, and the output should be the predicted shift and long shift.
            \item Define the loss function: Choose an appropriate loss function to measure the difference
            between the predicted shift and long shift and the actual shift and long shift.
            \item Train the model: Train the model using the training data and tune the hyperparameters to
            improve the performance of the model. Monitor the model's performance using the validation set.
            \item Evaluate the model: Evaluate the performance of the model on the testing set to see how well
            it generalizes to new data.
            \item Fine-tune the model: If the model does not perform well on the testing set, fine-tune the model by
            adjusting the architecture, hyperparameters, or training data.
            \item Use the model: Once we have a well-performing model, we can use it to predict the
            shift and long shift of new time series data.
        \end{enumerate}
        By following these steps, we can create a neural network for shift and long shift prediction.
        It's important to experiment with different architectures and hyperparameters to find the best
        model for our data.\\
        \\
        \textbf{Choose activation function}\\
        \\
        Choosing the activation function for a neural network for shift and long shift prediction depends on the
        architecture of the network and the problem being solved.\\
        \\
        For a recurrent neural network (RNN) architecture such as Long Short-Term Memory (LSTM) or
        Gated Recurrent Unit (GRU), the most commonly used activation function is the hyperbolic tangent (tanh) function.
        This is because the tanh function produces outputs in the range $[-1, 1]$, which makes it suitable for RNNs
        that use recurrent connections to store and update memory over time.\\
        \\
        However, other activation functions such as the Rectified Linear Unit (ReLU) or its variants, such as leaky
        ReLU or exponential ReLU, can also be used in RNNs for shift and long shift prediction. These activation
        functions are computationally efficient and can be used in deep neural networks without the problem of
        vanishing gradients.\\
        \\
        In general, the choice of activation function should be based on the characteristics of the problem being
        solved, the architecture of the network, and the properties of the activation function itself.
        Experimentation with different activation functions can help to determine the most appropriate one
        for a given problem.\\
        \\
        \textbf{Train neural site}\\
        \\
        To train a neural network for shift and long shift prediction, we can follow these steps:
        \begin{enumerate}
            \item Collect and preprocess the data: Collect the time series data that we want to use for training and
            testing the neural network. Preprocess the data by normalizing it and splitting it into training,
            validation, and testing sets.
            \item Define the architecture: Choose the architecture for our neural network based on the problem we
            want to solve. For shift and long shift prediction, a recurrent neural network (RNN) architecture
            such as a Long Short-Term Memory (LSTM) or Gated Recurrent Unit (GRU) may be suitable.
            \item Define the input and output: Define the input and output of our neural network. The input
            should be a sequence of data points, and the output should be the predicted shift and long shift.
            \item Define the loss function: Choose an appropriate loss function to measure the difference
            between the predicted shift and long shift and the actual shift and long shift. Mean Squared Error (MSE) is
            commonly used for regression problems such as this.
            \item Train the model: Train the model using the training data and tune the hyperparameters to improve the
            performance of the model. Use the validation set to monitor the model's performance and avoid overfitting.
            \item Evaluate the model: Evaluate the performance of the model on the testing set to see how well it
            generalizes to new data. Compare the predicted shift and long shift values with the actual values.
            \item Fine-tune the model: If the model does not perform well on the testing set, fine-tune the
            model by adjusting the architecture, hyperparameters, or training data.
            \item Use the model: Once we have a well-performing model, we can use it to predict the shift and long
            shift of new time series data.
        \end{enumerate}
        It's important to experiment with different architectures and hyperparameters to find the best model for
        our data. Additionally, it's important to properly preprocess the data and monitor the model's performance
        to avoid overfitting. With these steps, we can train a neural network for shift and long shift prediction.\\
        \\
        \textbf{Run neural site to get optimal shift and long shift}\\
        \\
        To run a trained neural network for shift and long shift prediction, we can follow these steps:
        \begin{enumerate}
            \item Preprocess the data: If we have new data that we want to use for prediction, preprocess it in the
            same way as the training data by normalizing it and formatting it into sequences.
            \item Load the trained model: Load the trained model into memory using a deep learning framework such as
            Tensorflow, PyTorch, or Keras.
            \item Prepare the input data: Prepare the input data by formatting it into sequences that match the input
            shape of the model.
            \item Make predictions: Use the predict function of the model to make predictions on the input data.
            The output of the model will be the predicted shift and long shift values.
            \item Postprocess the output: If necessary, postprocess the output of the model by denormalizing it or
            transforming it back into the original format.
            \item Evaluate the predictions: Evaluate the quality of the predicted shift and long shift values by
            comparing them with the actual shift and long shift values. Calculate performance metrics such as mean
            squared error or mean absolute error to quantify the accuracy of the predictions.
            \item Refine the model: If the predicted shift and long shift values are not accurate enough,
            refine the model by adjusting the architecture, hyperparameters, or training data and repeat the process.
        \end{enumerate}
        With these steps, we can run a trained neural network for shift and long shift prediction and obtain optimal
        shift and long shift values. It's important to note that the quality of the predictions depends on the
        quality and quantity of the data used for training and the architecture and hyperparameters of the model.
        Therefore, it's important to experiment with different models and training data to find the optimal solution.
    \section{Weights for each period} \label{sec:weights}
    To create a mathematical model for predicting sales data with periodical trends, we can use a seasonal
    ARIMA (SARIMA) model. This model takes into account seasonal variations in the data and uses autoregressive and
    moving average terms to capture the patterns and trends in the data.\\
    \\
    To create the weights for the SARIMA model, we can follow these steps:
    \begin{enumerate}
        \item Identify the seasonal period: Determine the seasonal period of the sales data. This can be done by
        analyzing the autocorrelation function (ACF) and partial autocorrelation function (PACF) of the data.
        \item Estimate the parameters: Estimate the parameters of the SARIMA model using the sales data.
        This can be done using statistical software such as R or Python.
        \item Interpret the weights: Once the parameters have been estimated, interpret the weights of the model
        to understand how they contribute to the prediction. The weights represent the strength of the relationship
        between the sales data and the predictor variables.
        \item Validate the model: Validate the model by comparing the predicted sales data with the actual sales data.
        Use statistical measures such as mean squared error or mean absolute error to evaluate the
        accuracy of the predictions.
        \item Refine the model: Refine the model by adjusting the weights or the parameters of the model based on
        the validation results. Repeat the validation process until the model provides accurate predictions.
    \end{enumerate}
    It's important to note that creating a SARIMA model can be complex and requires knowledge of time series
    analysis and statistical modeling. Therefore, it's recommended to seek the help of a data scientist
    or a statistician for assistance.
    \section{Extended long-term prediction} \label{sec:extlonglp}
    For my purpose i created extended long-term prediction which handle the seasonality and repeated patterns
    in economics data. This correction mechanism is used to reflect historical peaks in graphs thrue dataset.
    This simple correction increase accuracy of the model and get better response due tue psychology,
    sociology and marketing aspects in dataset.\\
    \\
    To set up periodical weights its needs to create a vector of correction parameters from original dataset using
    median and standard deviation statistical parameters from dataset. As a base equation we will user equation
    for long-term prediction\ref{eq:ltlp} with the new weights and get:
    \begin{equation} \label{eq:eltlp}
        \hat{y}(t) = \sum_{i=1}^{p} \beta_i y(t-i) * \gamma(t)
    \end{equation}
    \begin{itemize}
        \item $\hat{y}(t)$ is the predicted value of the time series at time t
        \item $y(t-i)$ is the value of the time series at time $t-i$ (i.e., p lags back)
        \item $\beta_1, \beta_2, \dots, \beta_p$ are the coefficients or weights assigned to the past values of the time series
        \item $\gamma(t)$ is the weight from weights vector for each position corresponding to dataset
    \end{itemize}
    When building a long-term prediction model, the use of weights in this use case have a significant effect on
    the model's performance. Weights are used to assign different levels of importance to different
    features or inputs in the model.\\
    \\
    In a long-term prediction model, the weight assigned to each input can determine its impact on the model's predictions.
    Using weights in order prediction model, the weight assigned to a historical company's financial data are more important
    than the weight assigned to its historical visitors. This is why we are ignore this data for prediction.\\
    \\
    If the weights are not carefully chosen, the model may give too much importance to some inputs and not enough to others,
    leading to inaccurate predictions. Therefore, selecting the right weights is crucial for creating an accurate and
    effective long-term prediction model. In our case it reduce the prediction error and returns more usable
    data for order prediction.\\
    \\
    In addition, the choice of weights can also affect the model's ability to adapt to changing conditions over time.
    A model with carefully chosen weights can be more resilient and adaptable to changes in the input data,
    while a model with poorly chosen weights may be less flexible and struggle to adjust to new information.
    So it's really important to set correct shift and period to recalculation of the shift.
    This is ride by neural site from \ref{subsec:neural}.\\
    \\
    Overall, the use of weights is an important consideration in building our predictive model, and careful selection
    can significantly improve the accuracy and effectiveness of our long-term prediction model.

    \section{Combining all principles to forecast process} \label{subsec:combining_models}
    \begin{enumerate}
        \item Run neural site to get right order of linear prediction \\
        \\
        In this section we have trained Neural site from section \ref{subsec:neural} and detect correct order from
        our neural site really fast.
        \item Run neural site to get righ shift and long shift of linear prediction\\
        Than ve can run our next neural site trained for detect shift for our prediction. In this neural site as
        one of the inputs we are using results from previous neural site.
        \item Run extended long-term prediction to ger predicted values of sales data\\
        When we have optimal order and shift we are able to run linear prediction in a standard way.
        \item Apply periodical wages to previous response data\\
        Results from previous step are corrected by the weights vector which is calculated over the trained dataset.
        \item Calculate comparison criteria for results and plot data and mean square errors\\
        When we run all the models we are able to calculate comparison criteria from \ref{subsec:comparison} and
        get a table of results to plot errors and compare predicted and measured orders from validation dataset.
    \end{enumerate}
